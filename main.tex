%% =========================== %%
%% LaTexについて
%% 適切な余白を採用するためにjsclassesを用いており
%% ここでは, jsbook(書籍用)を使用している.
%% そのため, reportというプロパティが必要になる.
%% =========================== %%
\documentclass[report, 11pt, a4paper]{jsbook}
\usepackage[utf8]{inputenc}

\usepackage{titlesec}
\usepackage{here, ascmac}
% 段落に空ける間隔で, 全角3文字分の指定. 2段組の場合に使用.
%\setlength{\columnsep}{3zw} 
\usepackage{fancyhdr} % ヘッダ・フッターのカスタマイズ
% -- 画像 -- %
\usepackage[dvipdfmx]{graphicx}%%画像読み込みに必要
\usepackage[dvipdfmx]{color}
\usepackage{here} % 画像を任意の位置に指定する
% -- テキスト -- %
\usepackage{color} 
\usepackage{ascmac} % テキストを線で括る
% -- 図・表 -- %
\usepackage{float} % 図を任意の位置に指定する
\usepackage{multirow} % 表内の複数行をまとめる
\usepackage{here}
\usepackage{array,booktabs} % テーブルに横罫線を引く


\title{卒業論文}
\author{加藤 雄大}
\date{January 2020}

\begin{document}

% 表紙
%% =========================== %%
\begin{titlepage}
\begin{center} %% 表紙なので中央寄せ
% 書体をゴシック体に設定
\textgt{
{\LARGE 卒業論文}\\
\vspace{40pt}
{\huge スマートウォッチを用いた長時間心拍データの活用と検討} % --ここにタイトル--
\vspace{50pt}
\begin{flushright}
{\Large 令和2年1月} % --論文作成日時--
\end{flushright}
\vspace{150pt}
% --自分の所属、学籍番号、氏名--
{\LARGE
インテリジェントシステム学科\\
学籍番号 644303\\
\vspace{10pt}
加藤 雄大\\
}
\vspace{120pt}
{\LARGE
京都産業大学 コンピュータ理工学部
}
}
\end{center}
\end{titlepage}
%% =========================== %%
\newpage


% 概要
%% =========================== %%
{\Large \textgt{概要}\\}
% 書体を明朝体に設定
\textmc{
\flushright % 左寄せ
{\fontsize{11pt}{15pt}\selectfont % フォントサイズと行送りを指定
 最終的に記述予定
}
}
%% =========================== %%
\newpage

% 目次
%% =========================== %%
{\large \tableofcontents}
%% =========================== %%
\clearpage

% 本文
%% =========================== %%
\chapter{背景}
\section{はじめに}
近年, ウェアラブルデバイスの市場は成長傾向にあり[1], 私たちの日常の多くの機会で目にするようになった. ウェアラブルデバイスとは,  その名前の通り「装着が可能な(=Wearable)」電子デバイスの総称である. 主なウェアラブルデバイスの種類として, メガネ型のスマートグラス, ゴーグル型のヘッドマウントディスプレイ(Head Mount Display:HMD), そして腕時計型のスマートウォッチが挙げられる. その中でも特に注目されているのが, スマートウォッチである. 以下の図☆はGoogle Trendsでウェアラブルデバイス,スマートフォン,スマートウォッチのそれぞれの検索頻度を比較した図である. 2014年の時点ではスマートウォッチが一般向けに普及が進んでいなかったため検索頻度が有意に低いことが理解できる. しかし, その後一般への普及が進み, 2020年1月ではスマートウォッチとスマートフォンとの検索頻度の差が埋まろうとしている. このことから一般の認知度としてスマートフォンとスマートウォッチは同等の存在になりつつあることが理解できる. また, ウェアラブルデバイスがほとんど検索されていないことからスマートウォッチがウェアラブルデバイスの一部としてではなく, スマートウォッチそのものとして認知されている傾向にあることが理解できる.\\
 そのスマートウォッチの代表的な製品とセンサ類をまとめた表☆を以下に示す. 表1.1は, IDCの2019年第1四半期のウェアラブルデバイスの市場規模の上位5位のデータ[☆]とオムロンのHeartGuideを参照したものである. センサ類は大きく分け, 体の姿勢を測定するためのセンサである加速度センサ・ジャイロセンサと生体データを収集するためのセンサである心拍センサ・心電図センサ・血圧センサに分けることができる. Apple Watchの心電図センサ, HeartGuideは血圧センサという先進的な例を除いて, 代表的なスマートウォッチおいて心拍センサが共通して搭載されていることが理解できる. (※Apple WatchはSerise4から心電図の測定が可能となっているが日本では薬事法により
利用が認められていない) このような傾向が見られる理由として, ヘルスケア需要の高まりが挙げられる. 昨今, 日本における重大な社会問題として国や医療従事者に対する医療負担の増大は少子高齢化の波を受け一層広まり続けている. また, 平均寿命も延び続けていることから, 生活の質を高めるためのQoL(Quality of Life)を高めようとする傾向が強い. その一端として, ヘルスケアや予防医学が注目されている. スマートウォッチに心拍センサが標準で備わっている理由は, そのためである. なので, 心拍センサを用いた具体的な事例として, 運動の質を定量的に図るために. 運動時の心拍数を計測したり, 運動の強度や運動用を心拍数を用いて計算するために用いられる. また, その活用法以外にも, 睡眠時には, 睡眠の質を数値化するために, 自律神経の活動を算出するためにも用いられる. さらには, 異常心拍の警告にも活用されている[☆].  \\
 このような活用事例から, 一見スマートウォッチの心拍データは有効に活用されているようにも見える. しかし, そのような心拍センサの利用法として, いずれも短い時間での運動, 睡眠, 安静時の変化量としてのみの活用に留まっているのが現状である. 本来, スマートウォッチのウェアラブルデバイスとしての強みである24時間の長期間に渡る測定結果は, 進行の緩やかな疾患や疾患の前兆による体調の変化(自律神経の変化)を捉えているはずだと私は考える. スマートウォッチの長期間測定心拍データが活用されることで, 将来的には, 診察や診断を医療機関ではなく24時間行えるため, 少子高齢化による医療従事者の負担や医療の地域格差も取り除くことができるはずである.
 そこで, 本研究ではスマートウォッチから測定された長期間心拍データを用いて様々な活用法の検討とプロトタイプの提案を行う.  

\begin{figure}[h]
\centering
\includegraphics[keepaspectratio, width=100mm]{1.eps}
\caption{Google Trendsによる検索頻度の比較}
\label{fig:goolge_trends}
\end{figure}


\begin{table}[H]
\centering
\caption{代表的なスマートウォッチとセンサの比較}
\begin{tabular}{ccccccc}
\hline
製品                   & 販売元     & 加速度 & ジャイロ & 心拍 & 心電図 & 血圧 \\ \hline
Apple Watch Series 5 & Apple   & ◯   & ◯    & ◯  & ◯   &    \\
Venu Granite         & Garmin  & ◯   & ◯    & ◯  &     &    \\
Fitbit Versa 2       & Fitbit  & ◯   &      & ◯  &     &    \\
Wena Wrist Active    & Sony    & ◯   &      & ◯  &     &    \\
Galaxy Watch         & Samsung & ◯   & ◯    & ◯  &     &    \\
HeartGuide           & OMRON   &     &      & ◯  &     & ◯  \\ \hline
\end{tabular}
\end{table}
 
\section{本論文の目的}
本研究では, スマートウォッチを用いて測定された長時間心拍データの様々な観点から有効性を検討することを目的としている. 

\section{本論文の意義}
近年, 様々な情報がビックデータとして扱われている中で, ヘルスケアや医療の分野でもビックデータとしての活躍が期待されている. しかし, 実際の医療現場で幅広く普及するまでに至っていないのが現実である. よって, 医療機関での導入より先に, 個人でのヘルスケアにおいて自身の長時間測定された生体データを分析・管理することが求められている.  
よって, 本研究は, 将来的な実現が求められる個人での生体データの管理と有用性の検討を行うことを目標としている. 


% 目次
%% =========================== %%
\chapter{関連研究}
本研究の位置付けを示すために, 以下に関連研究を挙げる. 第2.1章では, 主観的健康観の指標として24時間の心拍変動解析の有効性の検討について, 第2.2章では, ホルター心電図を用いた日常生活のモニタリングの意義について述べられたものである. 第2.3章で本研究の位置付けについてまとめた.

\section{24時間心拍変動と主観的健康観に関する研究}
心拍変動は, 自律神経の活動を評価するための指標として, ホルター心電図の開発とともに盛んに研究が行われてきた. 特に, ストレスケアなど社会全体の健康感が変化したことにより, 個人の健康の価値観の多様性が認められるようになった. そのため, ヘルスケアおいて, 心拍変動の活用を期待されている. しかし, 自律神経は精神的影響, 生体的影響など様々な因子により変化するため明確な活用法が発見されていないのが現状である. 
また, 日常において心拍変動を測定するには従来はホルター心電図やR-R間隔計測が用いられるため, 高価であり多数の被験者を測定することが困難であった.\\
~~これに対し, 大石らはPOLAR社製のハートレートモニターを用いて自律神経評価が人の健康の生理指標と成り得るかどうかの検討を行った. 手法として, RR間隔に対し周波数解析を行い, HF, LF, LF/HFというそれぞれ交感神経や副交感神経の活動を図るための指標と主観的健康感を図るためのVisual Analog Scale(VAS)による自作アンケートの結果を比較した. アンケートには, 身体的, 精神的, 社会的健康観, 幸福感のそれぞれ4項目を測定した. その結果として, 心拍変動によって得られた自律神経の活動と, VASの相関関係において多くの指標で優位な相関関係が観察された. 上記の研究により, 長時間の心拍変動を持ちいた1つの活用例が示された. 

\section{日常生活下の生体情報モニタリング:ホルター心電図から見える未来像}
2.1章で述べた通り心拍変動は多くの因子よる影響を受けているため長時間測定の意義を定量的に示すのは困難である. しかし, これから情報処理技術が発展する中で非侵襲的に生体情報を取得するセンサ類の種類や精度が向上することで影響する因子を1つずつ特定していくことが可能になるかもしれない. そして, その一歩目として不整脈などの心疾患の前兆を捉えるために使用されるホルター心電図を用いることで, 日常生活下での因子を特定するために早野は以下の研究に取り組んだ. 
早野の研究によると, 医学的な視点から見た日常化での生体モニタリングの科学的進歩は, それまで医療機関での診察で得られた1次元の情報の集まりから, 多次元の情報へ変化していることが提言されている. それは, 人の生活習慣・行動や社会環境と健康・疾患との関係について私たちの想像をはるかに超えた速度や範囲で拡大する可能性が示唆されていた. 上記の研究による検討は, これからの医療分野と情報分野の連携による可能性を示したものであり, 日常生活下の生体情報モニタリングの意義が示された例であった.

\section{本研究の位置付け}
2.1章では, スマートウォッチを用いた長時間の心拍データの活用例として, 24時間心拍変動と主観的健康観に関する研究を取り上げた. 上記の研究では, 心拍変動解析によって得られた自律神経の活動指標と主観的健康観の因果関係を評価することを目的とその有効性が示唆された. これは, 心拍データは様々な因子の影響を受けるため, このような関係性を示すことできる. 心拍データの複雑さ故に何かの観点に絞られければ, 研究としての位置付けをするのが困難であるためである.\\
~~また, 2.2章では, スマートウォッチではなく従来の心拍変動を捉えるために医療機関等で貸し出しが認められるホルター心電図を用い日常化での活用の検討が行われたものである. 早野の研究で述べられている通り, 心拍変動解析が医療にもたらした影響は大きく, それまで生体指標の揺らぎに医学的意義は認められなかった. しかし, ホルター心電図の登場により, 一定期間の連続的測定が可能になり, そのデータから変動の周波数の大きさが評価できるようになった. これは, スマートウォッチの登場によっても同様のことが言えると考える. これまでのホルター心電図やスパイダーフラッシュ心電図をはじめとする長時間生体データ測定器は高価でかつ軽量ではないため, 気軽に身につけることや日常生活に一定の行動の制限が起こり得た. それにより, データを取得できる対象は, 循環器に違和感を感じ医療機関に受診したものや診察により身につけることが提言されたもののみとなっていた. その対象から得られるデータは当然異常を含むもので, 健常者にはおおよそフォーカスされていなかった. \\
~~しかし, 昨今のスマートフォンの普及は, 上記の常識を覆す可能性であり, ヘルスケアやフィットネスを目的として購入した健常者は, 無意識的に長時間の生体データを記録し続ける. これは, 世界規模で行われ, 既存の心拍計測装置の比にならないペースで拡大している. そのような科学的資源を利用する価値はいうまでもないことだと私は考える. ただ, ここで述べておかないといけないことは, 心電図での心拍の測定とスマートウォッチでの心拍の測定法は異なるといったことである. 心電図は心臓の洞結節から発せられる電気信号を捉え, その振幅の最も大きいR波と連続するR波の間隔(RR間隔)から心拍を計算する方式である. 一方のスマートウォッチの測定法は, 光電式容積脈波記録法であり, これは心臓が鼓動したタイミングで伸縮する血管内に流れるヘモグロビンに対し, 赤外線を照射しその反射量(血流量)の変化を捉えることで心拍を計算する方式である. なので, 直接の動きを捉えている心電図の方が精度が高く, 光電式心拍計はそれに劣るのが自明の理である. 将来的にさらに光電式心拍計の精度が高まることを期待し, そのような状況で収集したデータの活用法の検討を行わなければない.\\

\chapter{事前調査}
スマートウォッチを用いた心拍の長時間データの有効性を検討するため, 事前調査としてfitbit inspire HRを用いて長期間(1週間)の心拍データを取得を行なった. 以下に, その詳細を示す.

\section{被験者}
被験者は, 男性1名(年齢22歳, 身長192.0cm 体重82.0kg)とした.\\
 測定は日曜日-土曜日の7日間(168時間)とした. 予備実験中の注意すべき事項として, スマートウォッチを外すタイミングは入浴時のみとし, 入浴時は可能な限り短い時間での入浴に制限した. これは, fitbit inspire HRで入浴が推奨されていなかったためである. また, 通学などの移動を除く, 過度な運動を行うことに制限を設けた. \\
 上記に挙げたこと以外の, 日常生活の行動(運動, 睡眠など)に制限を行わなかった. 

\section{測定機器・方法}
心拍数を測定するための機器として, fitbit社製のfitbit inspire HRを使用した. fitbit inspire HRでは24時間の心拍の測定が可能であり, 従来のスマートウォッチの心拍計と同様の光電式容積脈波記録法を用いられている. fitbit inpire HRより取得された心拍データは, 本体にも記録される他, 連携しているスマートウォッチを通じてfitbitのクラウド上にライフログとして記録される. 心拍データの他にも運動データ, 睡眠データも同様に記録される.\\
 利き手に装着した場合, ノイズがより多くなるため, 利き手ではない左手の手首に装着し測定を行なった. \\
 また, 測定後, 記録されたデータは, fitbitのクラウド上から1secごとのデータをして取得した. 

\section{仮説}
仮説として, 以下の事項が挙げられる. 

\begin{enumerate}
  \item 日毎に行動パターンが類似している場合, 心拍数の変動に一定の類似性が認められる.
  \item 心拍数の増減は, 日常の生活イベントと密接に同期している. 
  \item 心拍変動解析で行われる時間領域解析・周波数解析により心拍数変動においても同様の特徴量抽出を行うことが可能である.
\end{enumerate}

まず, 1つ目の仮説について, 人は社会に属する限り, 学校や会社など組織によって決められた時間, その組織に貢献しなければならない. そのため, 多くの一般人は, 週の大半が決められた時間によって行動が制限されている. そのような, 現代社会のシステム故に, 人の行動は一定時間パターン化できるのではないかと考える. そのパターン化された日々の行動は, 運動負荷がパターンごとに変化することが少ないと考えられるため, 心拍の変動に一定の類似性が認められると考えられる. また, 変動のみならず, 心拍数を解析した場合の平均や中央値などの結果も同様に類似しているのではないかと考えられる. \\
 次に, 2つ目の仮説について, 日常の生活イベントとは, 起床, 食事, 移動(それに伴う運動後), 入浴, 睡眠といった, 人間が生きていく上で年齢に関係なく発生するイベントである. いずれのイベントにおいても自律神経の活動に大きな影響を及ぼすイベントであるが, 長期間での測定でその変動が一律であるとは限らないため, 測定結果をこのような観点で確認することに意義がある. \\
 最後に, 3つ目の仮説について, 従来の自律神経活動の評価で用いられる時間領域解析・周波数解析による各指標と, 心拍数変動における各解析結果に因果関係があるかどうかを確認するものである. 心拍数の増減では, 進行の緩やかな長期間での変化を捉えることはできるに加え, 進行の早い, 短い期間での変化を捉えることができると, 疾患の予測・予知がスマートウォッチでも行えることの裏付けとなる. 
 
\section{結果}
結果として, 長期間計測心拍データの24時間ごとに心拍数変動を時系列データとしてプロットしたものが以下の図☆である. 
 
\begin{figure}[!h]
\centering
\includegraphics[width=140mm]{2.png}
\caption{長時間の心拍データ}
\label{fig:goolge_trends}
\end{figure}

図☆の青色の帯が睡眠時, 黄色の帯が移動時を表す. \\
 睡眠時において, 心拍数の増減は昼間活動時と逆転し, 心拍数が減少するといった明確な日内変動が確認された. これは, 通説の通りの副交感神経活動の増加と交感神経活動の低下によるものである. この変動は全ての日において確認ができた. 睡眠時間の長さ睡眠の時刻は日によって大きく異なっていたが, それに伴う睡眠時の心拍数に対する影響は見られなかった. \\
 移動時において, 徒歩や自転車など運動が発生する移動時には大きな心拍の上昇が見られ, 公共交通機関や自家用車の運転を行なっている時は同じ移動時においても心拍の上昇は見られなかった. このため, 同じ移動時の時間帯においても心拍の振幅の激しい箇所と落ち着いた箇所が混在していることが図から読み取れる. \\
 さらに, 日常生活でのイベントと心拍数変動が同期している時点として, 起床後の心拍の上昇と入浴後の心拍の上昇が見られた. これは, まず前者の睡眠では副交感神経の高まりが収まり, 交感神経の大きな高まりになっているためである. 後者の入浴後では, 入浴による血管拡張の作用や水圧の影響などにより心拍数の上昇が伴う[]. この心拍上昇の2つの要因は, 異常な心拍上昇の検出時には取り除かないといけないノイズとなる. \\
 異常な心拍上昇の検出数を曜日ごとにまとめた表☆を記す.  ここでの異常な心拍上昇は頻脈の基準である1分間の心拍数が100以上である状態の回数を検出した. 心拍のスパークとして確認できた振幅の数は少ないことが表から理解できるが, 実際の異常な心拍は6日目・金曜日の12:30から19:00までと7日目・土曜日の13:00から14:00までに数時間に渡る異常心拍の山が見られた. 前者は, 事務所に到着後の安静時なのだが, この心拍異常が起こる前にエナジードリンクを飲んでおり, その作用ではないかと考えられる. また, 後者については, 移動時に伴う運動後から心拍上昇が正常値に収まらなかった. このように, 表で示した, 一時的な心拍異常と数時間に渡って続く心拍異常の2パターンが存在することが理解できた. \\

\section{考察}
 仮説と結果について, 考察を行う. まず, 1つ目の仮説に対する結果として, 今回の予備実験では, 行動パターンと心拍数変動に有意な類似性が認められなかった. その理由として, 被験者の行動パターンの複雑さが挙げられる. 被験者は, 通学の際に, 複数の交通機関への乗り継ぎを行なっており, 移動の違いを考慮しなかったためである. 同じ, 最寄り駅までの移動においても, 心拍の上昇が見られる箇所と見られない箇所があるなど, 移動時に理想的な心拍変動の類似性は認められなかった. しかし, より行動のパターンが制限されている社会人などにおいては, 行動パターンと心拍数変動に類似性が認められる可能性がある.\\
 次に2つ目の仮説に対する結果として, 多くの日常のイベントと心拍数の増減は密接に同期していることが認められた. 特に, 最も増減が認められたイベントは, 起床後の心拍数の増加と入浴後の心拍数の増加である. しかし, 唯一, 食後の副交感神経活動の増加に伴う心拍の減少は認められなかった. これらの項目は, 対象者の心拍変動の定量化を行う上で加味しないといけない要素であることがこの結果から理解できる. \\
 最後に3つ目の仮説に対する結果は今後解析を行う予定である.


\begin{table}[]
\centering
\caption{異常な心拍上昇の検出数と日時}
\begin{tabular}{cccccc}
\hline
曜日  & 異常心拍の総数 & 安静時 & 睡眠中 & 睡眠後 & 入浴後 \\ \hline
日曜日 & 2       & 2   & 0   & 0   & 0   \\
月曜日 & 2       & 1   & 0   & 0   & 1   \\
火曜日 & 2       & 0   & 0   & 0   & 2   \\
水曜日 & 2       & 1   & 0   & 0   & 1   \\
木曜日 & 2       & 0   & 1   & 0   & 1   \\
金曜日 & 5       & 0   & 0   & 2   & 3   \\
土曜日 & 5       & 0   & 0   & 0   & 5   \\ \hline
\end{tabular}
\end{table}


% 提案手法
%% =========================== %%
\chapter{提案手法}
予備調査の結果を元に, 長期心拍データの活用を目指したプロトタイプ手法の提案を行う. 

\section{イベントの取得}
予備実験で得られた結果より, 安静時・運動時・睡眠時で大きく心拍変動の変容が異なるため, 3つの状態分けが必要となる. また, 各状態から遷移する前後(睡眠後, 運動後, 入浴後)は特にその変化量が大きくなるため, その変化を心拍変動の異常として捉えないようにするための工夫が必要である. 睡眠時と安静時は比較的, 連続性を持っているため状態の監視が比較的容易であるが, 運動時は, 特に例外的運動において突発的かつ短時間で行われるため監視が難しいと言える. 

\section{提案手法事項}
提案事項として, まず, 疾患の前兆となる緊急性の低い変化を捉えそれを注意喚起することが挙げられる. 長時間の心拍測定データでは, 従来のホルター心電図のような24時間のみのデータの変化量ではなく, 1週間/1ヶ月 /1年といったように柔軟に変化を捉えるレンジの調節ができる. 前述で述べたとおり, 緊急性の高い疾患の予知や予測は多くの研究がなされており, それらは, 100msec毎に起こる心拍変動を解析し, その短いレンジの中で行われてきた. しかし, それらの前兆となり得るメタボリックシンドロームを含む生活習慣病などの進行は非常に緩やかなものであるため, 短いレンジで捉えることはかなり厳しい. そのロングタームの測定を可能にしているのが, スマートウォッチの大きな役割の1つであるということが言えるのではないか.\\  

\section{プロトタイプのUI/UX}
以下にプロトタイプの画面についての解説を行う.

\begin{figure}[H]
\centering
\includegraphics[width=60mm]{3.png}
\caption{プロトタイプのUI/UX}
\label{fig:goolge_trends}
\end{figure}


% おわりに
%% =========================== %%
\chapter{おわりに}

\section{まとめ}
本研究では...
まとめは最後に書く

\section{課題}
スマートウォッチの長時間心拍データには, 様々な課題が存在する. まず, 最も大きな課題として, セキュリティの課題が挙げられる. 長時間心拍データの変動は, 事前実験の結果からも理解できる通り, 被験者の日常の行動から大きく影響を受けるものである. そのため, 被験者の長期間の行動パターンが第三者に渡ることは, 非常に危険であるため, 安全上大きな意味を持つ値になる. なので, 容易に第三者にデータ転送をすることは, 脅威となり得るが, 医療機関との連携を考えた際は, 容易なエクスポートが求められるものまた事実である. なので, 医療情報に関わるセキュリティの問題はこれからも留意しなければならない. \\
 次に, 心拍数の変動は年齢・身体・性別・運動歴/運動量によっても大きく変化するものであるため, その個人ごとに合わせた基準が必要になるという点である. 運動歴によっては, そもそものアスリートでは副交感神経トーンを表す相関が失われているという研究も行われている. また, 年齢とともに自律神経活動の恒常性が失われる. なので, 心拍数や心拍変動を用いた予知・予測において, 非常に重要であり, そのバイアスを考慮した活用が必要である.


\section{展望}
展望として, スマートウォッチの光電式心拍計の精度が心電図に匹敵する精度が担保されたとき, 初めてスマートウォッチを用いた自律神経評価や心疾患をはじめとする多くの疾患の予測・予知に用いられることだろうと考えられる. それにより, 人類は遂に身体に装着する自分専用の生体管理ツールを手に入れることになる. これまでの医療機関から支給されるものとは異なり私たちの生活に阻害をせず, ファッションの一部ともなっているスマートウォッチはより高い普及が進むことが予想できる. その予想される普及により, 私たちは, 診断・診察を常時行いながら, さらに多くの生体情報をそれ以後の代に渡って蓄え続けることになる. その貯蓄されていくデータは増え続けるあたり更により細かな生体の変化を捉え, さらに細かな予測・予知に用いられるだろう. 情報社会に相応しい未来であり, 情報資源という価値はこれからも生き続けることの裏付けでもある. \\





















\end{document}