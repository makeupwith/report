%% =========================== %%
%% LaTexについて
%% 適切な余白を採用するためにjsclassesを用いており
%% ここでは, jsbook(書籍用)を使用している.
%% そのため, reportというプロパティが必要になる.
%% =========================== %%
\documentclass[report, 11pt, a4paper]{jsbook}
\usepackage[utf8]{inputenc}

\usepackage{titlesec}
\usepackage{here, ascmac}
% 段落に空ける間隔で, 全角3文字分の指定. 2段組の場合に使用.
%\setlength{\columnsep}{3zw} 
\usepackage{fancyhdr} % ヘッダ・フッターのカスタマイズ
% -- 画像 -- %
\usepackage[dvipdfmx]{graphicx}%%画像読み込みに必要
\usepackage[dvipdfmx]{color}
\usepackage{here} % 画像を任意の位置に指定する
% -- テキスト -- %
\usepackage{color} 
\usepackage{ascmac} % テキストを線で括る
% -- 図・表 -- %
\usepackage{float} % 図を任意の位置に指定する
\usepackage{multirow} % 表内の複数行をまとめる
\usepackage{here}
\usepackage{array,booktabs} % テーブルに横罫線を引く


\title{卒業論文}
\author{加藤 雄大}
\date{January 2020}

\begin{document}

% 表紙
%% =========================== %%
\begin{titlepage}
\begin{center} %% 表紙なので中央寄せ
% 書体をゴシック体に設定
\textgt{
{\LARGE 卒業論文}\\
\vspace{40pt}
{\huge スマートウォッチを用いた長期間心拍データの活用法と有効性の検討} % --ここにタイトル--
\vspace{50pt}
\begin{flushright}
{\Large 令和2年1月} % --論文作成日時--
\end{flushright}
\vspace{150pt}
% --自分の所属、学籍番号、氏名--
{\LARGE
インテリジェントシステム学科\\
学籍番号 644303\\
\vspace{10pt}
加藤 雄大\\
}
\vspace{120pt}
{\LARGE
京都産業大学 コンピュータ理工学部
}
}
\end{center}
\end{titlepage}
%% =========================== %%
\newpage


% 概要
%% =========================== %%
{\Large \textgt{概要}\\}
% 書体を明朝体に設定
\textmc{
\flushright % 左寄せ
{\fontsize{11pt}{15pt}\selectfont % フォントサイズと行送りを指定
 最終的に記述予定
}
}
%% =========================== %%
\newpage

% 目次
%% =========================== %%
{\large \tableofcontents}
%% =========================== %%
\clearpage

% 本文
%% =========================== %%
\chapter{背景}
\section{はじめに}
近年, ウェアラブルデバイスの市場は成長傾向にあり[1], 私たちの日常の多くの機会で目にするようになった. ウェアラブルデバイスとは,  その名前の通り「装着が可能な(=Wearable)」電子デバイスの総称である. 主なウェアラブルデバイスの種類として, メガネ型のスマートグラス, ゴーグル型のヘッドマウントディスプレイ, そして腕時計型のスマートウォッチが挙げられる. その中でも特に昨今, 注目されているのが, スマートウォッチである. 以下の図☆はGoogle Trendsでウェアラブルデバイス,スマートフォン,スマートウォッチのそれぞれの検索頻度を比較した図である. 2014年の時点ではスマートウォッチの普及が進んでいなかったため検索頻度が有意に低いことが理解できる. しかし, その後一般への普及が進み, 2020年1月ではスマートウォッチとスマートフォンとの検索頻度の差が埋まろうとしている. このことから一般の認知度としてスマートフォンとスマートウォッチは同等の存在になりつつあることが理解できる. また, ウェアラブルデバイスがほとんど検索されていないことからスマートウォッチがウェアラブルデバイスの一部としてではなく, スマートウォッチそのものとして認知されている傾向にあることが分かる.\\
 そのスマートウォッチの代表的な製品とセンサ類をまとめた表☆を以下に示す. 表1.1は, IDCの2019年第1四半期のウェアラブルデバイスの市場規模の上位5位のデータ[☆]とオムロンのHeartGuideを参照したものである. スマートウォッチは身体に装着sルウという性質上m スマートフォンよりも多くのセンサが搭載されている. そのようなセンサ類は大きく分け, 体の姿勢を測定するためのセンサである加速度センサ・ジャイロセンサと生体データを収集するためのセンサである心拍センサ・心電図センサ・血圧センサに分けることができる. Apple Watchの心電図センサ, HeartGuideは血圧センサという先進的な例を除いて, 表☆より, 代表的なスマートウォッチおいて心拍センサが共通して搭載されていることが理解できる. (※Apple WatchはSerise4から心電図の測定が可能となっているが日本では薬事法により
利用が認められていない[]) このような傾向が見られる理由として, ヘルスケア需要の高まりが挙げられる. 昨今, 日本における重大な社会問題として国や医療従事者に対する医療負担の増大は少子高齢化の波を受け一層広まり続けている. また, 平均寿命も延び続けていることから, 生活の質を高めるためのQoL(Quality of Life)を高めようとする傾向が強い. これらのヘルスケア需要の高まりからスマートウォッチ の心拍センサに対する期待値は大きく, そのため多くのスマートウォッチに心拍センサが標準で備わっている. では, 実際にスマートウォッチの心拍センサはどのような活用がなされているのか活用事例を挙げていく. 活用事例にはApple Watchを参照して行う.\\
 まず, 1つ目の活用例として, 運動時の活用が挙げられる. Apple Watchでは, ワークアウトというアプリで運動時の心拍数の確認や心拍データを用いて計算された消費カロリーを確認することも可能である. 次に, 2つ目の活用例として, 睡眠時の活用が挙げられる. Apple Watchでは, 睡眠の質を睡眠時を心拍のデータなどを用いて測定を行い, レム睡眠, ノンレム睡眠がどの程度行われていたのかを自動で算出する機能を持っている. そして, 3つ目の活用例として, 異常な心拍のアラート機能としての活用がある. これは, 安静時に異常な心拍(高い心拍:120BPM以上, 低い心拍:40BPM)が10分間起こった場合にアラートを出すものである[☆]. 以上のような心拍センサを活用例はApple Watch以外のスマートウォッチにおいても同様の機能として備わっている. \\
 このような活用事例から, 一見スマートウォッチの心拍データは有効に活用されているようにも見える. しかし, 上記のような心拍センサの活用法は, いずれも短い時間での運動, 睡眠, 安静時の変化量を捉えたものである. 本来, スマートウォッチのウェアラブルデバイスとしての強みは24時間の長期間に渡る生体データの計測にあると考える. 既存の心拍データを日常下で計測するための医療機器であるホルター心電図は貸し出し期間がおおよそ1日と非常に短い[]. それに対し, スマートウォッチは充電の許す限り, 24時間何日間でも測定することが可能となっているのである. このスマートウォッチを用いた長期間心拍データは, スマートウォッチの普及と相まって情報資源として大変重要な価値を持つようになった. そのような膨大な情報資源を活用する上で, 本研究はその先駆けとなると考えている. そして, 将来的には, 診察や診断を医療機関ではなくとも24時間行うことができるようになり, その結果, 少子高齢化による医療従事者の負担や医療の地域格差も取り除くことができるはずである.
 そこで, 本研究ではスマートウォッチから測定された長期間心拍データを用いて様々な活用法と有用性の検討とプロトタイプの提案を行う.  

\begin{figure}[H]
\centering
\includegraphics[keepaspectratio, width=100mm]{1.eps}
\caption{Google Trendsによる検索頻度の比較}
\label{fig:goolge_trends}
\end{figure}


\begin{table}[H]
\centering
\caption{代表的なスマートウォッチとセンサの比較}
\begin{tabular}{ccccccc}
\hline
製品                   & 販売元     & 加速度 & ジャイロ & 心拍 & 心電図 & 血圧 \\ \hline
Apple Watch Series 5 & Apple   & ◯   & ◯    & ◯  & ◯   &    \\
Venu Granite         & Garmin  & ◯   & ◯    & ◯  &     &    \\
Fitbit Versa 2       & Fitbit  & ◯   &      & ◯  &     &    \\
Wena Wrist Active    & Sony    & ◯   &      & ◯  &     &    \\
Galaxy Watch         & Samsung & ◯   & ◯    & ◯  &     &    \\
HeartGuide           & OMRON   &     &      & ◯  &     & ◯  \\ \hline
\end{tabular}
\end{table}
 
\section{本論文の目的}
本研究では, スマートウォッチを用いて測定された長時間心拍データの様々な観点から有効性を検討することを目的としている. 

\section{本論文の意義}
近年, 様々な情報がビックデータとして扱われている中で, ヘルスケアや医療の分野でもビックデータとしての活躍が期待されている. しかし, 実際の医療現場で幅広く普及するまでに至っていないのが現実である. よって, 医療機関での導入より先に, 個人でのヘルスケアにおいて自身の長時間測定された生体データを分析・管理することが求められている.  
よって, 本研究は, 将来的な実現が求められる個人での生体データの管理と有用性の検討を行うことを目標としている. 


% 目次
%% =========================== %%
\chapter{関連研究}
本研究の位置付けを示すために, 以下に関連研究を挙げる. 第2.1章では, 主観的健康観の指標として24時間の心拍変動解析の有効性の検討について, 第2.2章では, ホルター心電図を用いた日常生活のモニタリングの意義について述べられたものである. 第2.3章で本研究の位置付けについてまとめた.

\section{24時間心拍変動と主観的健康観に関する研究}
心拍変動は, 自律神経の活動を評価するための指標として, ホルター心電図の開発とともに盛んに研究が行われてきた. 特に, ストレスケアなど社会全体の健康感が変化したことにより, 個人の健康の価値観の多様性が認められるようになった. そのため, ヘルスケアおいて, 心拍変動の活用を期待されている. しかし, 自律神経は精神的影響, 生体的影響など様々な因子により変化するため明確な活用法が発見されていないのが現状である. 
また, 日常において心拍変動を測定するには従来はホルター心電図やR-R間隔計測が用いられるため, 高価であり多数の被験者を測定することが困難であった.\\
~~これに対し, 大石らはPOLAR社製のハートレートモニターを用いて自律神経評価が人の健康の生理指標と成り得るかどうかの検討を行った. 手法として, RR間隔に対し周波数解析を行い, HF, LF, LF/HFというそれぞれ交感神経や副交感神経の活動を図るための指標と主観的健康感を図るためのVisual Analog Scale(VAS)による自作アンケートの結果を比較した. アンケートには, 身体的, 精神的, 社会的健康観, 幸福感のそれぞれ4項目を測定した. その結果として, 心拍変動によって得られた自律神経の活動と, VASの相関関係において多くの指標で優位な相関関係が観察された. 上記の研究により, 24時間の心拍変動を持ちいた1つの活用例が示された. 

\section{日常生活下の生体情報モニタリング:ホルター心電図から見える未来像}
心拍変動は多くの因子よる影響を受けているため長時間測定の意義を定量的に示すのは困難である. しかし, これから情報処理技術が発展する中で非侵襲的に生体情報を取得するセンサ類の種類や精度が向上することで影響する因子を1つずつ特定していくことが可能になるかもしれない. そして, その一歩目として不整脈などの心疾患の前兆を捉えるために使用されるホルター心電図を用いることで, 日常生活下での因子を特定するために早野は以下の研究に取り組んだ. 
早野の研究によると, 医学的な視点から見た日常化での生体モニタリングの科学的進歩は, それまで医療機関での診察で得られた1次元の情報の集まりから, 多次元の情報へ変化していることが提言されている. それは, 人の生活習慣・行動や社会環境と健康・疾患との関係について私たちの想像をはるかに超えた速度や範囲で拡大する可能性が示唆されていた. 上記の研究による検討は, これからの医療分野と情報分野の連携による可能性を示したものであり, 日常生活下の生体情報モニタリングの意義が示された例であった.

\section{心拍変動と臨床応用}
自律神経系は, 交感神経系と副交感神経系の2つの神経系に分類され, それぞれが拮抗しながら, 多くの内臓の臓器の管理を行っている. その中でも, 心拍数において, 心拍数の増加が交感神経の強い働きを示し, 心拍数の減少が副交感神経の強い働きを示すということが広く知られていたため, 古くから研究が盛んに行われてきた. そして, 90年代にホルター心電図が携帯型心電図の地位を確立したのち, さらに日常的な心拍変動の取得が容易となったことから研究が一層行われるようになった. 心拍変動の主な活用法として, 様々な疾患や前兆の予知・予測である. それは, 心臓そのものの疾患の予測を始め, 幅広い身体の異常を捉えることに成功した. その例についてを3つ述べる. \\
 まず,  不整脈に関して, その中でも特に急性の高い致死性不整脈について, 池主らは, 致死性不整脈の予知について臨床像から検査所見の観点をまとめた. その中に心拍変動を用いられた予知についての説明として, 心筋梗塞症例のうち, SDNN(RR間隔の標準分散)が50ms以下の罹患者にて, 死亡率が5.3倍高くなったと述べられていた. 次に, 無呼吸症候群について, 藤本は心拍変動解析の1つであるフラクタル解析を用いて, 睡眠時無呼吸症候群患者のLF値の異常なパワー減少が見られた. この異常な値は, 治療後の患者には見られなかったため, 睡眠時無呼吸症候群の指標としての可能性が示された. そして, 最後に, メタボリックシンドローム患者と心拍数の関係が示された河合らの研究では, 心拍数の4分位数で分類を行い, その結果, メタボリックシンドロームの危険因子となるBMI, 体脂肪, 血糖などを含む多くの値で頻度が有意に高いことが示された. これにより, 高心拍であることの危険性の裏付けが行われた.\\
 上記の研究の他にも, ストレス, 心筋虚血, 熱中症, てんかんなど多くの予知・予測が行われている. 私は, 疾患と緊急性の規則性が存在するという考えからそれらの先行研究との関係性について考えを示す. 図☆が, 疾患の緊急性を表したものである. 疾患の緊急度とその疾患の現在検査に用いられている機器を表している. スマートウォッチは生活習慣病対策からヘルスケアのため使用され, ホルター心電図は主に貸し出しが認められる対象が, 不整脈や狭心症の罹患者であり, 医療機関での心電図検査は, より緊急性の高い疾患の疑いのある対象に対しては, 心電図による検査が行われる. これに対して, 関連研究として, 挙げられる心拍変動解析を用いた疾患や体調の予知・予測は主にホルター心電図を用いて, より緊急性の高い例から低い例まで幅広く行われている. しかし, その多くが, 短期変動の分析に留まっている. 長期変動の分析は, ホルター心電図を用いた例では考慮されていない. しかし, 将来的に, スマートウォッチの心拍抽出の精度が担保された場合には, 初めて, 長期変動の分析について焦点が当てられることが期待される. そのような状況下で, 速やかな活用が行えるよう, 本研究では, 長期変動分析についての検討を行う. 

\begin{figure}[H]
\centering
\includegraphics[width=140mm]{urgency_class.png}
\caption{疾患と緊急性}
\label{fig:goolge_trends}
\end{figure}


\section{本研究の位置付け}
2.1章では, スマートウォッチを用いた長時間の心拍データの活用例として, 24時間心拍変動と主観的健康観に関する研究を取り上げた. 上記の研究では, 心拍変動解析によって得られた自律神経の活動指標と主観的健康観の因果関係を評価することを目的とその有効性が示唆された. これは, 心拍データは様々な因子の影響を受けるため, このような関係性を示すことできる. 心拍データの複雑さ故に何かの観点に絞られければ, 研究としての位置付けをするのが困難であるためである.\\
~~また, 2.2章では, スマートウォッチではなく従来の心拍変動を捉えるために医療機関等で貸し出しが認められるホルター心電図を用い日常化での活用の検討が行われたものである. 早野の研究で述べられている通り, 心拍変動解析が医療にもたらした影響は大きく, それまで生体指標の揺らぎに医学的意義は認められなかった. しかし, ホルター心電図の登場により, 一定期間の連続的測定が可能になり, そのデータから変動の周波数の大きさが評価できるようになった. これは, スマートウォッチの登場によっても同様のことが言えると考える. これまでのホルター心電図やスパイダーフラッシュ心電図をはじめとする長時間生体データ測定器は高価でかつ軽量ではないため, 気軽に身につけることや日常生活に一定の行動の制限が起こり得た. それにより, データを取得できる対象は, 循環器に違和感を感じ医療機関に受診したものや診察により身につけることが提言されたもののみとなっていた. その対象から得られるデータは当然異常を含むもので, 健常者にはおおよそフォーカスされていなかった. \\
~~しかし, 昨今のスマートフォンの普及は, 上記の常識を覆す可能性であり, ヘルスケアやフィットネスを目的として購入した健常者は, 無意識的に長時間の生体データを記録し続ける. これは, 世界規模で行われ, 既存の心拍計測装置の比にならないペースで拡大している. そのような科学的資源を利用する価値はいうまでもないことだと私は考える. ただ, ここで述べておかないといけないことは, 心電図での心拍の測定とスマートウォッチでの心拍の測定法は異なるといったことである. 心電図は心臓の洞結節から発せられる電気信号を捉え, その振幅の最も大きいR波と連続するR波の間隔(RR間隔)から心拍を計算する方式である. 一方のスマートウォッチの測定法は, 光電式容積脈波記録法であり, これは心臓が鼓動したタイミングで伸縮する血管内に流れるヘモグロビンに対し, 赤外線を照射しその反射量(血流量)の変化を捉えることで心拍を計算する方式である. なので, 直接の動きを捉えている心電図の方が精度が高く, 光電式心拍計はそれに劣るのが自明の理である. 将来的にさらに光電式心拍計の精度が高まることを期待し, そのような状況で収集したデータの活用法の検討を行わなければない.\\


% 3. 事前調査
%% =========================== %%
\chapter{事前調査}
スマートウォッチを用いた心拍の長期間データがどのような特徴を有しているのか, それをどのように活用できるのかをを検討するため, 事前調査としてfitbit inspire HRを用いて長期間(1週間)の心拍データを取得を行なった. 以下に, その詳細を示す.

\section{被験者}
被験者は, 大学生1名(年齢22歳, 男性, 身長192.1cm 体重80.4kg)とした.\\
 測定は日曜日-土曜日の7日間(168時間)とした. 事前調査の注意すべき事項として, スマートウォッチを外すタイミングは入浴時のみとし, 入浴時は可能な限り短い時間での入浴に制限した. これは, fitbit inspire HRで入浴が推奨されていなかったためである. また, 通学などの移動を除く, 過度な運動を行わないように制限を設けた. \\
 上記に挙げたこと以外の, 日常生活の行動(運動, 睡眠など)に制限を行わなかった. 

\section{測定機器・測定方法}
心拍数を測定するためのスマートウォッチとして, fitbit社製のfitbit inspire HRを使用した. fitbit inspire HRでは24時間の心拍の測定が可能である. 従来のスマートウォッチの心拍計と同様の光電式容積脈波記録法を用いられている. fitbit inpire HRより取得された心拍データは, 本体にも記録される他, 連携しているスマートフォンを通じてfitbitのクラウド上にライフログとして記録される. 心拍データの他にも運動データ, 睡眠データも同様に記録される.\\
 装着を行うのは, 利き手ではない左手の手首に行った. これは, 利き手に装着した場合, ノイズがより多くなるためである. \\
 また, 測定後, 記録されたデータは, fitbitのクラウド上からAPIからPythonによって心拍データを取得した. 

\section{データの解析}
取得された心拍データは, プロットする際には, 視覚的に捉えやすいようにするため, 移動平均法による平滑化を行なった. 移動平均窓は30である. それ以外のデータでは, 1sec, 1min, 15minとそれぞれ適切なスケールで解析を行った. 以下に, 取得したデータに対して行った解析手法を挙げた.

\begin{enumerate}
  \item データの特性値
  \item 異常な心拍の検出
  \item 心拍変動解析の応用
\end{enumerate}

 まず, 1つ目のデータの特性値に関しては, 最も簡易な方法で, 一日毎の心拍の平均・中央値・分散・標準偏差を行った. これは, メタボリックシンドロームと心拍変動に関する関係性が述べられた河合らの論文[]にて評価指標として用いられていたため, 本事前調査でもデータの比較の対象とした. \\
 次に, 異常な心拍の検出だが, ここでは, 平滑化されたデータを対象にして行った. また, Apple Watchの異常心拍の閾値は, 通常の不整脈(頻脈, 徐脈)よりさらに大きな値としていることで, より緊急性の高いもののみに絞っているようだが, 本実験においては, 通常の不整脈の基準である, 頻脈:100BMP以上, 徐脈:40-50BPMを基準とした. \\
 最後の心拍変動解析の応用は, 従来の心拍変動解析がスマートウォッチの心拍数の変動にも活用できないかという検討のために, 行う. 心拍変動解析とは, 第2.3章で述べたとおり洞結節より発せられる電気信号の最も大きなピークR波とR波の間隔(RR間隔)の時系列データの揺らぎ(正常洞調律)を解析し, 自律神経活動の評価などを行うためのに用いられる指標である. 心拍変動解析の手法は, 時間領域解析と周波数解析の2つに分けることが出来る. 時間領域解析は, 心拍変動の特性値(平均, 標準偏差)を様々な観点から評価するものである. 一方の周波数解析は, 交感神経や副交感神経の明確な指標となる. そのような指標が, 心拍数の変動においても同様の指標となるかどうかの検討を行う.  心拍変動解析が, スマートウォッチから得られた心拍数変動にも活用できた場合, スマートウォッチを用いて疾患・その前兆の予知・予測に用いることが可能であるということの証明にもなる. 

\section{仮説}
仮説として, 先にいくつか提言しておかなければならない事項から述べる. \\
 まず, 

仮説として, 以下の3つの事項が挙げられる. 

\begin{enumerate}
  \item 日毎に行動パターンが類似している場合, 心拍数の変動に一定の類似性が認められる.
  \item 心拍数の増減は, 日常の生活イベントと密接に同期している. 
  \item 心拍変動解析で行われる時間領域解析・周波数解析により心拍数変動においても同様の特徴量抽出を行うことが可能である.
\end{enumerate}

まず, 1つ目の仮説について, 人は社会に属する限り, 学校や会社など組織によって決められた時間, その組織に貢献しなければならない. そのため, 多くの一般人は, 週の大半が決められた時間によって行動が制限されている. そのような, 現代社会のシステム故に, 人の行動は一定時間パターン化できるのではないかと考えた. そのパターン化された日々の行動は, 運動負荷がパターンごとに変化することが少ないと考えられるため, 心拍の変動に一定の類似性が認められると考えられる. また, 変動のみならず, 心拍数を解析した場合の平均や中央値などの結果も同様に類似しているのではないかと考えられる. \\
 次に, 2つ目の仮説について, 日常の生活イベントとは, 起床, 食事, 移動(それに伴う運動後), 入浴, 睡眠といった, 人間が生きていく上で年齢に関係なく発生するイベントである. いずれのイベントにおいても自律神経の活動に大きな影響を及ぼすイベントであるが, 長期間での測定でその変動が一律であるとは限らないため, 測定結果をこのような観点で確認することに意義がある. \\
 最後に, 3つ目の仮説について, 従来の自律神経活動の評価で用いられる時間領域解析・周波数解析による各指標と, 心拍数変動における各解析結果に因果関係があるかどうかを確認するものである. 心拍数の増減では, 進行の緩やかな長期間での変化を捉えることはできるに加え, 進行の早い, 短い期間での変化を捉えることができると, 疾患の予測・予知がスマートウォッチでも行えることの裏付けとなる. 
 
\section{結果・考察}
結果として, まず, 長期間計測心拍データの24時間ごとに心拍数変動を時系列データとしてプロットしたものが以下の図☆である. 図☆の青色の帯が睡眠時, 黄色の帯が移動時, 赤色の横線が安静時の平均心拍数を表している. プロットされた図から読み取れる事柄として, 睡眠時において, 心拍数の増減は昼間活動時と逆転し, 心拍数が減少するといった明確な日内変動が確認された. これは, 通説の通りの副交感神経活動の増加と交感神経活動の低下によるものである. この変動は全ての日において確認ができた. また, 睡眠時間の長さや睡眠の開始時刻は日によって大きく異なっていたが, それに伴う睡眠時の心拍数に対する影響は見られなかった. 同様にして, 移動時において, 徒歩や自転車など運動が発生する移動時には大きな心拍の上昇が見られ, 公共交通機関や自家用車の運転を行なっている時は同じ移動時においても心拍の上昇は見られなかった. このため, 同じ移動時の時間帯においても心拍の振幅の激しい箇所と落ち着いた箇所が混在していることが図から読み取れる. \\
 被験者の行動パターンに大きな規則性がなかったため, 仮説1で挙げたような心拍変動の類似性は認められなかった. しかし, 特性値での評価においては, 類似性が見られた. 表☆に, 各日にちごとの心拍数の平均, 中央値, 分散, 標準分散, そしてその該当する日に主にどのような行動をしているたのかを表した. これらのデータの類似度を測るため, ピアソンの積率相関係数を用いた. すると, ほぼ全てのデータにおいて強い相関が見られたが, その中でも最も強い相関が見られたのが, 6日目と7日目でその値は, 0.999734にも及んだ. 次に相関が大きかったのが1日目と3日目で値は, 0.998972であった. そして, 三番目に相関が大きかったのが, 4日目と5日目であった. 以上の類似性の非常に高かった上位3位まで日で行われた行動は全て同じであったことから, 1日の行動パターンの類似性の指標として, 特性値を用いることが有効である可能性が示せた.  
 
\begin{figure}[H]
\centering
\includegraphics[width=140mm]{1days.png}
\caption{1日目の心拍数の変動}
\includegraphics[width=140mm]{2days.png}
\caption{2日目の心拍数の変動}
\includegraphics[width=140mm]{3days.png}
\caption{3日目の心拍数の変動}
\label{fig:goolge_trends}
\end{figure}

\begin{figure}[H]
\centering
\includegraphics[width=140mm]{4days.png}
\caption{4日目の心拍数の変動}
\includegraphics[width=140mm]{5days.png}
\caption{5日目の心拍数の変動}
\includegraphics[width=140mm]{6days.png}
\caption{6日目の心拍数の変動}
\label{fig:goolge_trends}
\end{figure}

\begin{figure}[H]
\centering
\includegraphics[width=140mm]{7days.png}
\caption{7日目の心拍数の変動}
\label{fig:goolge_trends}
\end{figure}

\begin{table}[H]
\centering
\caption{1日毎の心拍の特性値と主な行動}
\begin{tabular}{cccccc}
\hline
日  & 平均    & 中央値   & 分散     & 標準分散  & 行動   \\ \hline
1日目 & 65.10 & 65.00 & 124.55 & 11.16 & 自宅   \\
2日目 & 69.39 & 70.00 & 121.27 & 11.01 & 自宅   \\
3日目 & 69.39 & 70.00 & 142.60 & 11.94 & 自宅   \\
4日目 & 77.64 & 75.00 & 236.59 & 15.38 & 大学   \\
5日目 & 76.80 & 75.00 & 282.51 & 16.81 & 大学   \\
6日目 & 84.85 & 84.00 & 566.62 & 23.80 & サークル \\
7日目 & 74.26 & 70.00 & 409.53 & 20.24 & サークル \\ \hline
\end{tabular}
\end{table}

 次に, 移動時を除く, 異常な心拍の検出結果を表☆に示す. \\
 今回検出できた異常な心拍のうち, 生理的な心拍上昇として, 起床後の心拍の上昇と入浴後の心拍の上昇が見られた. これは, まず前者の睡眠では副交感神経の高まりが収まり, 交感神経の大きな高まりになっているためである. 後者の入浴後では, 入浴による血管拡張の作用や水圧の影響などにより心拍数の上昇が伴う[]. この心拍上昇の2つの要因は, 健常者でも起こり得るため異常な心拍上昇の検出時には取り除かないといけないノイズとなる. \\
 上記の要因以外の異常な心拍数は, 安静時に多く見られた. 

\begin{table}[H]
\centering
\caption{異常な心拍上昇の検出数と日時}
\begin{tabular}{ccccccc}
\hline
日  & 異常心拍の総数 & 起床後 & 安静時 & 睡眠中 & 睡眠後 & 入浴後 \\ \hline
1日目 & 2       & 0   & 1   & 0   & 0   & 1   \\
2日目 & 2       & 0   & 1   & 0   & 0   & 1   \\
3日目 & 2       & 0   & 0   & 0   & 0   & 2   \\
4日目 & 2       & 0   & 1   & 0   & 0   & 1   \\
5日目 & 2       & 1   & 0   & 1   & 0   & 1   \\
6日目 & 4       & 2   & 0   & 0   & 0   & 1   \\
7日目 & 4       & 0   & 0   & 0   & 0   & 4   \\ \hline
\end{tabular}
\end{table}

 また, 異常な心拍な上昇以外にも, 異常な心拍が数時間に渡って続く山のような波形が見られた. そのような波形は, 6日目・金曜日の12:30から19:00までの6時間半と7日目・土曜日の13:00から14:00までに1時間の間で見られた. 前者は, 事務所に到着後の安静時なのだが, この心拍異常が起こる前にエナジードリンクを飲んでおり, その作用ではないかと考えられる. また, 後者については, 移動時に伴う運動後から心拍上昇が正常値に収まらなかった. このように, 表で示した, 一時的な心拍異常と数時間に渡って続く心拍異常の2パターンが存在することがわかった.\\

 事前調査のまとめとして, 行動パターンと心拍数の類似性は特性値を用いた相関係数から認められた. このことによって, メタボリックシンドロームなどの生活習慣病の緩やかな循環器系の異常として, 特性値を用いた警告が行えるのではないかという可能が示された. また, 異常な心拍数については, 睡眠後と入浴後のノイズを取り除くことで精度が高められるということ, 心拍の異常にはブロック状の異常も存在することがなどが明らかになった.
 

% 4. 提案手法
%% =========================== %%
\chapter{提案手法}
事前調査の結果を元に, 長期心拍データの活用を目指した提案手法の検討を行う. 

\section{長期変動分析による活用}
生活習慣病やメタボリックシンドロームといった疾患は, 進行が穏やかなため, 長期間に渡る変化を捉えなければ予知には至らない. しかし, 心拍変動は様々な生理的因子による影響を受けること, また長期間でデータを捉える場合には本人の活動による影響も考慮しなければならない. 今回の事前調査からは, 1日の心拍変動が睡眠時, 移動時, 安静時において大きく傾向が異なることが分かった. そのため, 長期変動分析において, それぞれの変容の隔たりと考慮するために以下の手法が適切であると考える.

\begin{enumerate}
  \item 睡眠時・移動時・安静時ごとの分析・比較
  \item 睡眠時・移動時・安静時の類似度が高いものをパターン分けし, 分析・比較
\end{enumerate}

 まず, 前者に関しては, それぞれの状態でのみ発生する前兆を捉えるために用いられる. 例えば, 睡眠時無呼吸症候群の場合, 発生する睡眠時にターゲットが絞れるといったものである. さらに, 移動時の分析・比較では, 突発的な運動に起因する疾患や前兆の予測に繋がると考えられる. 睡眠時無呼吸症候群のように, 1日の中で, 発生するタイミングが既知である場合は有効な手段であるが, それ以外には, 有効性が薄いと考えられる.
 次に, 後者について, 睡眠時・移動時・安静時の時間の割合や, 移動時においては行動パターンなどの類似性が高いものをパターン分けし, それぞれで比較するといったものである. 類似性に関しては, 1日ごとの特性値が有効であり, 平日・休日を比較する場合にも用いることが可能であると考える. そこから, 大凡同じパターンの日毎の類似性を長期に渡り比較することで, 変化するはずの要因が存在しないのにも関わらず, 類似度が異なっていた場合には, 何かしらの異常が生じていることが分かる. 

\section{異常な心拍上昇の経過観察}


\section{日常のイベントと心拍数}
予備実験で得られた心拍数のプロット図より, 安静時・運動時・睡眠時で大きく心拍変動の変容が異なるため, 3つの状態分けが必要となるということが理解できた. また, 各状態から遷移する前後(睡眠後, 運動後, 入浴後)は特にその変化量が大きくなるため, その変化を心拍変動の異常として捉えないようにするための工夫が必要である. 睡眠時と安静時は比較的, 連続性を持っているため状態の監視が比較的容易であるが, 運動時は, 特に例外的運動において突発的かつ短時間で行われるため監視が難しいことも考慮しなければならない. よって, 必要な値は, それぞれの状態での平均・中央値・分散・標準分散の特性値である. それらの状態での1日単位の値をさらに日毎に比較することで定量化が可能になるのではないかと考える.  

\section{異常心拍の検出}
先行事例として, 挙げられていた異常心拍の検出だが, 実際に事前調査で1週間の心拍数変動から新たな異常心拍の形として, 数時間に及ぶものを2度に渡り検出が行えた. 
これは, 推測される要因が実際に正しいかどうかは分からないが, 異常に心拍には様々な状態があるため, 一定の閾値を設けてしまうと, 本当に致死性の高い不整脈なのかどうかが不明である. なので, 異常心拍の検出においては, より高精度でかつ, 個人ごとにバイアスをかけて検出を行わないといけない,

提案事項として, まず, 疾患の前兆となる緊急性の低い変化を捉えそれを注意喚起することが挙げられる. 長時間の心拍測定データでは, 従来のホルター心電図のような24時間のみのデータの変化量ではなく, 1週間/1ヶ月 /1年といったように柔軟に変化を捉えるレンジの調節ができる. 前述で述べたとおり, 緊急性の高い疾患の予知や予測は多くの研究がなされており, それらは, 100msec毎に起こる心拍変動を解析し, その短いレンジの中で行われてきた. しかし, それらの前兆となり得るメタボリックシンドロームを含む生活習慣病などの進行は非常に緩やかなものであるため, 短いレンジで捉えることはかなり厳しい. そのロングタームの測定を可能にしているのが, スマートウォッチの大きな役割の1つであるということが言えるのではないか.\\  


% おわりに
%% =========================== %%
\chapter{おわりに}

\section{まとめ}
本研究では...
まとめは最後に書く

\section{課題}
スマートウォッチの長時間心拍データには, 様々な課題が存在する. まず, 最も大きな課題として, セキュリティの課題が挙げられる. 長時間心拍データの変動は, 事前実験の結果からも理解できる通り, 被験者の日常の行動から大きく影響を受けるものである. そのため, 被験者の長期間の行動パターンが第三者に渡ることは, 非常に危険であるため, 安全上大きな意味を持つ値になる. なので, 容易に第三者にデータ転送をすることは, 脅威となり得るが, 医療機関との連携を考えた際は, 容易なエクスポートが求められるものまた事実である. なので, 医療情報に関わるセキュリティの問題はこれからも留意しなければならない. \\
 次に, 心拍数の変動は年齢・身体・性別・運動歴/運動量によっても大きく変化するものであるため, その個人ごとに合わせた基準が必要になるという点である. 運動歴によっては, そもそものアスリートでは副交感神経トーンを表す相関が失われているという研究も行われている. また, 年齢とともに自律神経活動の恒常性が失われる. なので, 心拍数や心拍変動を用いた予知・予測において, 非常に重要であり, そのバイアスを考慮した活用が必要である.


\section{展望}
展望として, スマートウォッチの光電式心拍計の精度が心電図に匹敵する精度が担保されたとき, 初めてスマートウォッチを用いた自律神経評価や心疾患をはじめとする多くの疾患の予測・予知に用いられることだろうと考えられる. それにより, 人類は遂に身体に装着する自分専用の生体管理ツールを手に入れることになる. これまでの医療機関から支給されるものとは異なり私たちの生活に阻害をせず, ファッションの一部ともなっているスマートウォッチはより高い普及が進むことが予想できる. その予想される普及により, 私たちは, 診断・診察を常時行いながら, さらに多くの生体情報をそれ以後の代に渡って蓄え続けることになる. その貯蓄されていくデータは増え続けるあたり更により細かな生体の変化を捉え, さらに細かな予測・予知に用いられるだろう. 情報社会に相応しい未来であり, 情報資源という価値はこれからも生き続けることの裏付けでもある. \\





















\end{document}